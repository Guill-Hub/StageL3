\documentclass[a4paper, english, 10pt]{article}

\usepackage[utf8]{inputenc}
\usepackage[T1]{fontenc}

\usepackage{lmodern}
\usepackage{tgpagella}

\usepackage{geometry}
\geometry{ hmargin=3cm, vmargin=3.5cm }

\usepackage{fancyhdr}
\pagestyle{fancy}

\makeatletter
\newcommand{\@templatetitle}{On Strategyproof Picking Sequences} % Le titre
\newcommand{\@templatesubtitle}{~}
\newcommand{\@templateauthor}{Sylvain B.} % Les auteurs
\newcommand{\@templatedate}{\today} % Les auteurs
\newcommand{\@templatekeywords}{} % Les mots-clefs
\newcommand{\@templatesubject}{} % Le sujet
\makeatother

\makeatletter
\fancyhf{} % clear all fields
\fancyhead[C]{\nouppercase{\fontfamily{\sfdefault}\fontseries{m}\fontshape{sf}\footnotesize\selectfont\color{gray} \@templatetitle}} % Insérer le titre ici
\renewcommand{\headrule}{\color{lightgray}\hrule}
\fancyfoot[C]{\fontfamily{\sfdefault}\fontseries{m}\fontshape{sf}\small\selectfont\thepage}
\makeatother

\usepackage{natbib}
\renewcommand{\cite}[1]{\citep{#1}}
\usepackage{babel}

\setlength{\overfullrule}{10pt}

\usepackage{amsmath}
\usepackage{amssymb}
\usepackage{amsfonts}
\usepackage{latexsym}                   % Symboles
\usepackage{mathrsfs}                   % Oh, les jolies lettres calligraphiées !
\usepackage{stmaryrd}                   % Pour avoir les délimiteurs entiers
\usepackage[amsmath, thmmarks]{ntheorem} % Un environnement theorem custom
\usepackage{breakcites}
\usepackage{stmaryrd}
\usepackage{booktabs}
\usepackage{listings} % Pour insérer du code source


\usepackage{tcolorbox,listings}
\usepackage{fullpage}
\usepackage{color}

\definecolor{darkWhite}{rgb}{0.94,0.94,0.94}

\lstset{
	backgroundcolor=\color{darkWhite},
	breakatwhitespace=false,
	breaklines=true,
	captionpos=b,
	commentstyle=\color{red},
	deletekeywords={...},
	escapeinside={\%*}{*)},
	extendedchars=true,
	keepspaces=true,
	keywordstyle=\color{blue},
	language=C,
	morekeywords={*,...},
	showspaces=false,
	showstringspaces=false,
	showtabs=false,
	stepnumber=1,
	stringstyle=\color{gray},
	tabsize=4,
	title=\lstname,
}

\lstdefinestyle{frameStyle}{
	basicstyle=\footnotesize,
	numbers=left,
	numbersep=20pt,
	numberstyle=\tiny\color{black}
}

\tcbuselibrary{listings,skins,breakable}

\newtcblisting{customFrame}{
	arc=0mm,
	top=0mm,
	bottom=0mm,
	left=3mm,
	right=0mm,
	width=\textwidth,
	listing only,
	listing options={style=frameStyle},
	breakable
}


\usepackage{tikz}

\usepackage[pdftex]{hyperref}

\usepackage{theorem}
\theoremsymbol{$\blacksquare$}
{\theorembodyfont{\rmfamily}\newtheorem{exemple}{Exemple}}

\newcommand{\fixme}[1]{{\small\color{red}[FIXME : #1]}}
\makeatletter
\newcommand{\Omit}[1]{\ifhmode \@bsphack \@esphack \fi}
\makeatother

\definecolor{mycyan}{rgb}{0,0.2,0.4}
\hypersetup{%
  colorlinks=true,
  citecolor=mycyan,
  linkcolor=mycyan,
  anchorcolor=mycyan,
  urlcolor=mycyan,
  pdfstartview=FitH,
  pdfdisplaydoctitle=true,
  %pdfwindowui=false,
  pdftoolbar=true,
  pdfmenubar=true
}

\makeatletter
\hypersetup{%
  pdftitle={\@templatetitle},
  pdfauthor={\@templateauthor},
  pdfsubject={\@templatesubject},
  pdfkeywords={\@templatekeywords}
}
\makeatother

%%%%%%%%%%%%%%%%%%%%%%%%%%%%%%%%%%%%%%%%%%%%%%%%%%%%%%%%%%%%%%%%%%%%%%%%%%%%%%%%%%%%%%%%%%%
% Notations

\newcommand{\ee}{\mathcal{E}}
\newcommand{\var}[1]{\mathbf{#1}}
\newenvironment{resume}{
  \textbf{Abstract}\quad\itshape
}{
  \vskip\parskip
}
\newcommand{\motsclefs}[1]{\textbf{Keywords} \quad #1\\\vskip\parskip}

\DeclareMathOperator{\argmax}{argmax}
\newtheorem{definition}{Definition}
\newtheorem{theorem}{Theorem}

%
%%%%%%%%%%%%%%%%%%%%%%%%%%%%%%%%%%%%%%%%%%%%%%%%%%%%%%%%%%%%%%%%%%%%%%%%%%%%%%%%%%%%%%%%%%%





\makeatletter
\title{\@templatetitle}
\author{\@templateauthor}
\date{\@templatedate}
\makeatother

\makeatletter
\renewcommand{\maketitle}{
  \begin{center}
    {\LARGE\sffamily\bfseries \@templatetitle}\\\hrulefill\\
    {\large \@templatesubtitle}\\\vskip0.5cm
    \@templateauthor\\
    \@templatedate\\\vskip0.5cm
  \end{center}
}
\makeatother

\begin{document}
\renewcommand{\labelitemi}{\tikz\fill[rounded corners=1pt] (0, 0) -- (4pt, 2.5pt) -- (0pt, 5pt) -- cycle;}

\thispagestyle{plain}

\maketitle

\begin{abstract}
  This short note is about strategyproof picking sequences, namely,
  picking sequences where the agents pick all their objects in a row,
  in a serial dictatorship fashion. Hopefully, the computation of the
  fairest sequence is much easier in this case.
\end{abstract}


\section{Two agents}

We assume that there are two agents Alice and Bob ($A$ and $B$) and
$m$ objects labeled from 1 to $m$. We consider that the agents have
Borda utility functions. We assume Full Independence assumption: the
benevolent dictator is ignorant of Alice's and Bob's actual
preferences, but assumes that all the ordinal are equally probable.

We focus on picking policies of the following form:
\[
  \pi^k = \underbrace{AA \dots A}_{k \text{ times}}\underbrace{BB \dots B}_{m - k \text{ times}}
\]

The question is the following one. What is the integer $k$ such that
$\pi^k$ maximizes the egalitarian expected utility? In other words, we
look for:
\[
  \argmax_{k \in \{ 1, ... m \}}(\min(\overline{u(A, \pi^k)}, \overline{u(B, \pi^k)}))
\]

Of course, we assume that both Alice and Bob are rational players:
they act according to their dominant strategy. Here, the only dominant
strategy for both of them is to pick their preferred items (truthful
strategy).

\subsection{Computation of expected utility}

\subsubsection{Alice}

\label{AliceTh}
Alice's expected utility can be computed immediately. No matter what
Bob's preferences are she will always pick her $k$ best objects. Hence:
\[
  \overline{u(A, \pi^k)} = \frac{m(m+1)}{2} - \frac{(m-k)(m-k+1)}{2}
\]

We will assume without loss of generality, that Bob has the following
preferences over items: $1 \succ 2 \succ ... \succ m$. Concerning
Alice, all the linear orders on $\{ 1, ..., m \}$ are equally probable
(Full Independence assumption).

\subsubsection{Bob}
\label{BobTh}

Bob's expected utility is less immediate to compute. Let us introduce the following
notation:

\begin{definition}
  Let $n$, $k \leq n$ and $t \leq \frac{n(n+1)}{2}$ be three
  integers. Then:
  \[
    S(n, k, t) = \left|\{\mathcal{S} \subset \{1, \dots, n\}\ |\ \sum_{x_i \in \mathcal{S}} x_i = t \text{ and } |\mathcal{S}| = k\}\right|.
  \]
  In other words, $S(n, k, t)$ is the number of different subsets of
  $k$ elements from $\{1, \dots, n\}$ that sum up to $t$.
\end{definition}


The possible utilities for Bob range from $\frac{(n-k)(n-k+1)}{2}$ (if
the $k$ items picked by Alice are Bob's $k$ top ones) to
$\frac{n(n+1)}{2} - \frac{k(k+1)}{2}$ (if the $k$ items picked by
Alice are Bob's $k$ bottom ones). More generally, Bob will get utility
$U$ if and only if the items picked by Alice are worth
$\frac{n(n+1)}{2} - U$ for Bob. There are 
$S(n, k, \frac{n(n+1)}{2} - U)$ subsets of that kind. 
For every of those subset, the order between the k elements Alice will choose and the n-k that left (i.e Bob will get) doesn't matter
so any permutation of it will lead to the same utility U for Bob,  there is $ k! \times (n-k)! $ such permutations .
So Alice could choose $S(n, k, \frac{n(n+1)}{2} - U) \times k! \times (n-k)!$ linear orders verifying this property among the $n!$ possible ones. \
Then by definition of expected utility which (faire la jolie formule ) ... \
 $\overline{u(B, \pi^k)} =
{\sum_{U = \frac{(n-k)(n-k+1)}{2} }^{\frac{n(n+1)}{2} - \frac{k(k+1)}{2}} U \times Pr(U)} = {\sum_{U = \frac{(n-k)(n-k+1)}{2} }^{\frac{n(n+1)}{2} - \frac{k(k+1)}{2}} U \times \frac{ S(n, k, \frac{n(n+1)}{2} - U) \times k! \times (n-k)!}{n!}} $
.


\subsubsection{Compute(S(n,k,t))}
\begin{customFrame}
def subset_sum(n, k, target, memo=None):
	""" Computes (and returns) the number of possibilities of taking k different integers between 1 and n who sum up to target. """
	
	if memo is None:
		return subset_sum(n, k, target, [[[-1 for _ in range(target + 1)]
										for _ in range(k + 1)] for _ in range(n + 1)])
	if k == 0:
		return int(target == 0)
	elif target <= 0 or n == 0:
		return 0
	if memo[n][k][target] == -1:
		memo[n][k][target] = (
			subset_sum(n - 1, k - 1, target - n, memo) + 
			subset_sum(n - 1, k, target, memo))
	return memo[n][k][target]

\end{customFrame}

\begin{description}
	\item[Termination:] The natural n is decremented at each call of subset-sum, because $\mathbb{N}$ is well-ordered the algorithm end.
	\item[Correctness:] By induction over (n,k,t) (with lexicographic order):
	
	
		\begin{description}
			\item[Init :]
			\begin{description}
			\item[If memo is None:] Is called once, to pre-initialize the table used for memoization
			\item[If $k==0$ :] A subset of zero element is unique and can only sum up to 0, so $S(n,0,0) = 1$ and $S(n,0,t) = 0$ for $t > 0$ 
			\item[If $target <= 0$ :] Every utilitie is a strictly positive integer so it is impossible to reach a negative target
			\item[If $n==0$ :] There is no subset of the empty set
			\end{description}
			\item[Inductive step :] Let suppose that subset\_sum(n',k',target',memo) is egal to S(n',k',target') for every (n',k',target') < (n,k,target) then:
			 If $ memo(n,k,target) == -1 $ (i.e if subset\_sum(n,k,target,memo) hasn't be called before) then there is two distincts cases:
			  \begin{itemize}
			  	\item Alice pick the object of utility n, so she had to pick another k-1 objects from $\{1, \dots, n-1\}$ such as it sum up to t' = target - n, which is exactly S(n-1,k-1,target - n, memo) so subset\_sum(n-1,k-1,target - n, memo) by hypothese of induction
			  	\item Alice don't pick the object of utility n, so she has still k objects summing to target to choose from $\{1, \dots, n-1\}$, which is subset\_sum(n-1,k,target , memo) by hypothese of induction
			  \end{itemize}
		  Finally subset\_sum(n,k,t) = S(n - 1, k - 1, target - n, memo) + S(n - 1, k, target, memo)) = S(n,k,t)
		  We store that value in memo not to compute it every time we need it, and we return it
		  

			
			
			 
		\end{description}
\end{description} 

\subsubsection{Computing bob expexted utility}

	\begin{customFrame}
	def bobs_expected_utility(n, k, memo=None):
	"""
	Computes (and returns) the expected utility for Bob, assuming
	that there are n objects, and Alice picks the first k ones.
	"""
	
	total_utility = int(n * (n + 1) / 2)
	min_picked = int(k * (k + 1) / 2)
	max_picked = total_utility - int((n - k) * (n - k + 1) / 2)
	return sum((total_utility - picked) * subset_sum(n, k, picked, memo)
	for picked in range(min_picked, max_picked + 1)) * factorial(k) * factorial(n - k) / factorial(n)		
	\end{customFrame}
\label{BobComp}
\begin{theorem}
	bobs\_expected\_utility return $ \overline{u(B, \pi^k)} $ as a the direct application of \ref{BobTh}
\end{theorem}

\subsection{Optimal policy}

\subsubsection{Compute optimal policy}

	\begin{customFrame}
def optimal_cut(n):
"""
Computes (and returns) the optimal cut for n object. In other words,
this number is the integer k that guarantees the best egalitarian expected
utility for Alice and Bob if Alice picks the first k items and Bob the n-k
remaining ones.
"""
total_utility = int(n * (n + 1) / 2)
last_min = 0
memo = -np.ones((n + 1, total_utility + 1, total_utility + 1), dtype=np.float)
for k in range(1, n + 1):
alice = total_utility - int((n - k) * (n - k + 1) / 2)
bob = bobs_expected_utility(n, k, memo)
if alice > bob:
if last_min > min(alice, bob):
k-=1
alice = total_utility - int((n - k) * (n - k + 1) / 2)
bob = bobs_expected_utility(n, k, memo)
return (k , alice, bob)
last_min = min(alice, bob)
assert False, "We shouldn't be here..."
return
	
\end{customFrame}



\begin{theorem}[Correctness] \ 
	
	$\bullet$
	First we notice that we have the invariants loop :
	alice = $\overline{u(A, \pi^k)}$ and bob = $ \overline{u(B, \pi^k)} $ as a direct result of \ref{AliceTh} and \ref{BobComp}.
	
	
	$\bullet$ For k = 0 (alice pick nothing) there is $\overline{u(A, \pi^k)} < \overline{u(B, \pi^k)} $
	
	$\bullet$ For k = n (alice pick everything) there is  $\overline{u(A, \pi^k)} > \overline{u(B, \pi^k)} $
	
	$\bullet$ The application $\phi(k) \rightarrow \overline{u(A, \pi^k)} $ is strictly increasing (more k is high more Alice will have)
	
	$\bullet$ The application $\psi(k) \rightarrow \overline{u(B, \pi^k)} $ is strictly decreasing (more k is high less Bob will have)
	
	So the first time alice > bob is the breakpoint from where before, $\overline{u(A, \pi^k)} < \overline{u(B, \pi^k)} $ and after $\overline{u(A, \pi^k)} > \overline{u(B, \pi^k)} $ ie.
	$ \min(\overline{u(A, \pi^k)} ,\overline{u(B, \pi^k)} ) = \overline{u(A, \pi^k)}$ before and 
	$ \min(\overline{u(A, \pi^k)} ,\overline{u(B, \pi^k)} ) = \overline{u(B, \pi^k)} $ after.
	By growth of $\phi$ and decrease of $\psi$ the max of min is reached at this point, or the time before. It is why the algorithm check which of the min are max and return the argument of the max of the mins, and the expected utiliy of alice and bob for this argument
	

\end{theorem}

\subsubsection{Some numbers}

\begin{tabular}{lll}
  \toprule
  $m$ & $\hat{k}$ & $\hat{k} / m$\\
  \midrule
  10 & 4 & 0.4\\
  20 & 8 & 0.4\\
  30 & 12 & 0.4\\
  40 & 15 & 0.375\\
  50 & 19 & 0.38\\
  60 & 23 & 0.38333...\\
  100 & 38 & 0.38\\
  \bottomrule
\end{tabular}



% \bibliographystyle{plainnat-fr}
% \bibliography{partage}


\end{document}

%%% Local Variables:
%%% mode: latex
%%% TeX-master: t
%%% End:
